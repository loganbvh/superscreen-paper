\documentclass{article} % This command is used to set the type of document you are working on such as an article, book, or presenation

\usepackage{geometry} % This package allows the editing of the page layout
\usepackage{amsmath}  % This package allows the use of a large range of mathematical formula, commands, and symbols

\title{Highlights}
\date{\today}

\begin{document}
\maketitle
\begin{itemize}
    \item{SuperScreen is an open-source Python package that solves the London equation to simulate the Meissner response of two-dimensional superconductors for any value of the effective magnetic penetration depth.}
    \item{The package provides an intuitive object-oriented interface for the creation of models with complex geometries, and for post-processing and visualization of simulation results.}
    \item{The package can be used to calculate self- and mutual inductance in thin film superconducting devices.}
    \item{SuperScreen can be used to model flux trapping and fluxoid quantization effects, and calculate the supercurrent distribution due to Pearl vortices in two-dimensional superconductors.}
    \item{SuperScreen enables quantitative modeling of the supercurrent distribution in mesoscopic superconducting devices to help inform and interpret measurements.}
\end{itemize}
\end{document}