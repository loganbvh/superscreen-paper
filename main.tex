\documentclass{article}
\usepackage[utf8]{inputenc}
\usepackage{geometry}
\geometry{letterpaper, margin=1.5in}
\usepackage{amsmath, amssymb, bm}
\usepackage{minted}

\newcommand{\SuperScreen}{\mintinline{python}{SuperScreen}\,}

\title{\texttt{SuperScreen}}
\author{Logan Bishop-Van Horn}
\date{\today}

\begin{document}

\maketitle

\section{Introduction}
\label{section:introduction}

\SuperScreen is an open-source Python package developed to simulate the magnetic response of structures composed of one or more layers containing superconducting thin films. \SuperScreen solves the coupled Maxwell's and London's equations in and around superconducting films with spatially-varying penetration depth in the presence of applied magnetic fields and trapped flux using a matrix inversion method introduced by Brandt [ref.] and initially implemented by Kirtley [ref, ref] to model the magnetic response of scanning superconducting quantum interference device (SQUID) sensors.

In Section~\ref{section:model} we outline Brandt's model, and in Section~\ref{section:implementation} we describe its numerical implementation. In Section~\ref{section:overview}, we provide an overview of the structure of the \SuperScreen package, highlighting important functions, classes, and features, and discuss some important development details. In Section~\ref{section:examples}, we demonstrate how to set up and solve several types of models using \SuperScreen. Finally, in Section~\ref{section:conlusion} we discuss potential applications, extensions, and improvements to the package.

\section{The Model}
\label{section:model}

The goal of \SuperScreen is to model the magnetic response of a thin superconducting film, or a structure composed of multiple superconducting films (which may or may not lie in the same plane), to an applied inhomogeneous out-of-plane magnetic field
$H_{z,\,\mathrm{applied}}(x, y, z)$.

Given $H_{z,\,\mathrm{applied}}(x, y, z)$ and information about the geometry and magnetic penetration depth of all films in a superconducting structure, we aim to calculate the thickness-integrated current density (or sheet current) $\vec{J}(x, y)$ at all points inside the films, from which one can calculate the vector magnetic field $\vec{H}(x, y, z)$ at all points both inside and outside the films.

A convenient method for solving this problem was introduced in [Brandt-PRB-2005], and used in [Kirtley-RSI-2016] and [Kirtley-SST-2016] to model the magnetic response of scanning Superconducting Quantum Interference Device (SQUID) susceptometers.

In the London model of superconductivity, the magnetic field $\vec{H}(\vec{r})$ and 3D current density $\vec{j}(\vec{r})$ in a superconductor with London penetration depth $\lambda(\vec{r})$ obey the second London equation:
$\nabla\times\vec{j}(\vec{r})=-\vec{H}(\vec{r})/\lambda^2(\vec{r})$, where
$\nabla=\left(\frac{\partial}{\partial x}, \frac{\partial}{\partial y}, \frac{\partial}{\partial z}\right)$.

Brandt's model assumes that the current density $\vec{j}$ is approximately independent of $z$, such $\vec{j}(\vec{r}) = \vec{j}(x, y, z)\approx\vec{j}_{z_0}(x, y)$ for a film lying parallel to the $x-y$ plane at vertical position $z_0$. Working now with the thickness-integrated current density (sheet current) $\vec{J}(x, y)=\vec{j}_{z_0}(x, y)\cdot d$, where $d$
is the thickness of the film, the second London equation reduces to

\begin{equation}
    \label{eq:london}
    \nabla\times\vec{J}(x, y)=-\vec{H}(x, y)/\Lambda(x, y),
\end{equation}

where $\Lambda(x, y)=\lambda^2(x, y)/d$ is the effective penetration depth
of the superconducting film (equal to half the Pearl length [Pearl-APL-1964]).

It is important to note that the assumption $\vec{j}(x, y, z)\approx\vec{j}_{z_0}(x, y)$ is valid for only films that are thinner than their London penetration depth ($d<\lambda$, such that $\Lambda=\lambda^2/d>\lambda$). However the model has been applied with some success in structures with $\lambda\lesssim d$ [Kirtley-RSI-2016] [Kirtley-SST-2016]. Aside from this limitation, the method described below can in principle to used to model films with any effective penetration depth $0\leq\Lambda<\infty$.

Because the sheet current has zero divergence in the superconducting film ($\nabla\cdot\vec{J}=0$)
except at small contacts where current can be injected, one can express the sheet current in terms
of a scalar potential $g(x, y)$, called the stream function:

\begin{equation}
    \label{eq:stream}
    \vec{J}(x, y) = -\hat{z}\times\nabla g
    = \nabla\times(g\hat{z})
    = \left(\frac{\partial g}{\partial y}, -\frac{\partial g}{\partial x}\right).
\end{equation}

The stream function $g$ can be thought of as the local magnetization of the film, or the area density of magnetic dipole sources (see [Brandt] for more interesting properties of the stream function). We can re-write Eq.~\ref{eq:london} for a 2D film in terms of $g$:

\begin{align}
    \label{eq:london_stream}
    \begin{split}
        \vec{H}(x, y) &= -\Lambda\left[\nabla\times\vec{J}(x, y)\right]\\
        &= -\Lambda\left[\nabla\times\left(\nabla\times(g\hat{z})\right)\right]\\
        &= -\Lambda\left[\nabla(\nabla\cdot(g\hat{z}))-\nabla^2(g\hat{z})\right]\\
        &=\Lambda\nabla^2g(x,y)\hat{z}\\
        &=H_z(x, y)\hat{z}
    \end{split}
\end{align}
where $\nabla^2=\nabla\cdot\nabla$ is the Laplace operator. (The fourth line follows from the fact that $\nabla\cdot\left(g(x,y)\hat{z}\right) = 0$). From Ampere's Law, the three components of the magnetic field at position $\vec{r}=(x, y, z)$ due to a sheet of current lying in the $x-y$ plane (at height $z'$) with stream function $g(x', y')$ are given by:

\begin{align}
    \label{eq:field_from_kernel}
    \begin{split}
        H_x(\vec{r}) &= \int_S Q_x(\vec{r},\vec{r}')g(x', y')\,\mathrm{d}^2r'\\
        H_y(\vec{r}) &= \int_S Q_y(\vec{r},\vec{r}')g(x', y')\,\mathrm{d}^2r'\\
        H_z(\vec{r}) &= H_{z,\,\mathrm{applied}}(\vec{r})
        + \int_S Q_z(\vec{r},\vec{r}')g(x', y')\,\mathrm{d}^2r'  
    \end{split}
\end{align}

Here we assume an applied magnetic field $\vec{H}_\mathrm{applied}(\vec{r}')=H_\mathrm{z,\,\mathrm{applied}}(\vec{r}')\hat{z}$. $S$ is the film area (with $g = 0$ outside of the film), and $Q_x(\vec{r},\vec{r}')$, $Q_y(\vec{r},\vec{r}')$, and $Q_z(\vec{r},\vec{r}')$ are dipole kernel functions which give the relevant component of the magnetic field at position $\vec{r}=(x, y, z)$ due to a dipole of unit strength at position $\vec{r}'=(x', y', z')$:

\begin{align}
    \label{eq:kernels}
    \begin{split}
        Q_x(\vec{r}, \vec{r}') &=  3\Delta z\frac{x-x'}
        {4\pi[(\Delta z)^2+\rho^2]^{5/2}}\\
        Q_y(\vec{r}, \vec{r}') &=  3\Delta z\frac{y-y'}
        {4\pi[(\Delta z)^2+\rho^2]^{5/2}}\\
        Q_z(\vec{r}, \vec{r}') &=  \frac{2(\Delta z)^2-\rho^2}
        {4\pi[(\Delta z)^2+\rho^2]^{5/2}},
    \end{split}
\end{align}
where $\Delta z = z - z'$ and $\rho=\sqrt{(x-x')^2 + (y-y')^2}$.

Comparing Eq.~\ref{eq:london_stream} and Eq.~\ref{eq:field_from_kernel}, we have in the plane of the film:

\begin{align}
    \label{eq:integral_equation}
    \begin{split}
        \underbrace{H_z(\vec{r}) = \vec{H}(\vec{r})\cdot\hat{z}
        = \Lambda\nabla^2g(x, y)}_{z-\text{component of the total field}}
        = \underbrace{H_{z,\,\mathrm{applied}}(\vec{r})}_{\text{applied field}}
        + \underbrace{\int_S Q_z(\vec{r},\vec{r}')g(\vec{r}')\,\mathrm{d}^2r'}_{\text{screening field}},
    \end{split}
\end{align}
(where now $\vec{r}$ and $\vec{r}'$ are 2D vectors, i.e. $\Delta z=0$, since the film is in the same plane as itself). From Eq.~\ref{eq:integral_equation}, we arrive at an integral equation relating the stream function $g$ for points inside the superconductor to the applied field $H_{z,\,\mathrm{applied}}$:

\begin{equation}
    \label{eq:applied_to_stream}
    H_{z,\,\mathrm{applied}}(\vec{r})
    = -\int_S\left[
        Q_z(\vec{r},\vec{r}')-\delta(\vec{r}-\vec{r}')\Lambda(\vec{r}')\nabla^2\right
    ]g(\vec{r}')\,\mathrm{d}^2r'
\end{equation}

The goal, then, is to solve (invert) Eq.~\ref{eq:applied_to_stream} for a given $H_{z,\,\mathrm{applied}}$ and film geometry $S$ to obtain $g$ for all points inside the film (with $g=0$ enforced outside the film). Once $g(\vec{r})$ is known, the full vector magnetic field $\vec{H}(\vec{r})$ can be calculated at any point $\vec{r}$
from Eqs.~\ref{eq:field_from_kernel} and \ref{eq:kernels}.

\subsection{Films with holes}
\label{section:model:holes}

In films that have holes (regions of vacuum completely surrounded by superconductor), each hole $k$ can contain a trapped flux $\Phi_k$, with an associated circulating current $I_{\mathrm{circ},\,k}$. The (hypothetical) applied field that would cause such a circulating current is given by Eq.~\ref{eq:applied_to_stream} if we set $g(\vec{r})=I_{\mathrm{circ},\,k}$ for all points $\vec{r}$ lying inside hole $k$:

\begin{equation}
    \label{eq:Heff}
    H_{z,\,\mathrm{eff},\,k}(\vec{r}) = -\int_{\mathrm{hole}\,k}[
        Q_z(\vec{r},\vec{r}')-\Lambda(\vec{r}')\nabla^2
    ] I_{\mathrm{circ},\,k} \,\mathrm{d}^2r'.   
\end{equation}

In this case, we modify the left-hand side of Eq.~\ref{eq:applied_to_stream} as follows:

\begin{equation}
    \label{eq:Heff_sub}
    H_{z,\,\mathrm{applied}}(\vec{r}) - \sum_k H_{z,\,\mathrm{eff},\,k}(\vec{r})
    = -\int_S\left[
        Q_z(\vec{r},\vec{r}')-\delta(\vec{r}-\vec{r}')\Lambda(\vec{r}')\nabla^2\right
    ]g(\vec{r}')\,\mathrm{d}^2r'.
\end{equation}

\subsection{Multi-layer structures}
\label{section:model:multilayer}

For structures with multiple films lying in different planes or layers, with layer $\ell$ lying in the plane $z=z_\ell$,
the stream functions and fields for all layers can be computed self-consistently using the following recipe:

\begin{enumerate}
    \item{
        Calculate the stream function $g_\ell(\vec{r})$ for each layer $\ell$ by solving Eq.~\ref{eq:Heff_sub} given an applied field $H_{z,\,\mathrm{applied}}(\vec{r}, z_\ell)$.
    }
    \item{
        For each layer $\ell$, calculate the $z$-component of the field due to the currents in all other layers $m\neq\ell$ (encoded in the stream function $g_m(\vec{r})$)
        using Eq.~\ref{eq:field_from_kernel}.
    }
    \item{
        Re-solve Eq.~\ref{eq:Heff_sub} taking the new applied field at each layer to be the original applied field plus the sum of screening fields from all other layers. This is accomplished via the substitution
        $$
            H_{z,\,\mathrm{applied}}(\vec{r}, z_\ell) \to
            H_{z,\,\mathrm{applied}}(\vec{r}, z_\ell)
            + \sum_{m\neq\ell}
            \int_S Q_z(\vec{r},\vec{r}')g_m(\vec{r}')\,\mathrm{d}^2r.
        $$
    }
    \item{
        Repeat steps 1-3 until the solution converges.
    }
\end{enumerate}

Convergence can be quantified by, for example, calculating the total magnetic flux though all films and holes in the model at the end of each iteration.

\section{Numerical Implementation}
\label{section:implementation}

In order to numerically solve Eq.~\ref{eq:field_from_kernel} and Eq.~\ref{eq:Heff_sub}, we have to discretize the films, holes, and the vacuum regions surrounding them. We use a triangular
(Delaunay) mesh, consisting of $n$ points (or vertices)
which together form $m$ triangles.

Below we denote column vectors and matrices using bold font. $\mathbf{A}\mathbf{B}$
denotes matrix multiplication, with $(\mathbf{A}\mathbf{B})_{ij}=\sum_{k=1}^\ell A_{ik}B_{kj}$
($\ell$ being the number of columns in $\mathbf{A}$ and the number of rows in $\mathbf{B}$). Column vectors are treated as matrices with $\ell$ rows and 1 column. On the other hand, we denote element-wise multiplication with a dot: $(\mathbf{A}\cdot\mathbf{B})_{ij}=A_{ij}B_{ij}$ for two matrices
and $(\mathbf{A}\cdot\mathbf{v})_{ij}=(\mathbf{v}\cdot\mathbf{A})_{ij}=A_{ij}v_{i}$ for a matrix $\mathbf{A}$ and a column vector $\mathbf{v}$.

The matrix version of Eq.~\ref{eq:field_from_kernel} is:

\begin{equation}
    \label{eq:field_from_kernel_num}
    \underbrace{\mathbf{h}_z}_\text{total field}
    = \underbrace{\mathbf{h}_{z,\,\mathrm{applied}}}_\text{applied field}
    + \underbrace{(\mathbf{Q}\cdot\mathbf{w})\mathbf{g}}_\text{screening field}.
\end{equation}

The kernel matrix $\mathbf{Q}$ and weight matrix $\mathbf{w}$ together play the role of the
kernel function $Q_z(\vec{r},\vec{r}')$ for all points lying in the plane of the film. They are both $n\times n$ matrices determined solely by the geometry of the mesh.
$\mathbf{h}_z$, $\mathbf{h}_{z,\,\mathrm{applied}}$, and $\mathbf{g}$ are all $n\times 1$ vectors, with each row representing the value of the quantity at the
corresponding vertex in the mesh. There are several different methods for constructing the weight matrix $\mathbf{w}$, which are discussed in Section[] below. The kernel
matrix $\mathbf{Q}$ is defined in terms of a matrix with
$(\mathbf{q})_{ij} = \left(4\pi|\vec{r}_i-\vec{r}_j|^3\right)^{-1}$
(which is $\lim_{\Delta z\to 0}Q_z(\vec{r},\vec{r}')$ cf. Eq.~\ref{eq:kernels}),
and a vector $\mathbf{C}$:

\begin{equation}
    \label{eq:kernel_matrix}
    Q_{ij} = (\delta_{ij}-1)q_{ij}
    + \delta_{ij}\frac{1}{w_{ij}}\left(C_i + \sum_{l\neq i}q_{il}w_{il}\right),
\end{equation}
where $\delta_{ij}$ is the Kronecker delta function. The diagonal terms involving $\mathbf{C}$ are meant to work around the fact that $q_{ii}$ diverge (see [Brandt-PRB-2005] for more details), and the vector is defined as

\begin{equation}
    \label{eq:C_vector}
    C_i = \frac{1}{4\pi}\sum_{p,q=\pm1}\sqrt{[\Delta x - p(x_i-\bar{x})]^{-2} + [\Delta y - q(y_i-\bar{y})]^{-2}},
\end{equation}
where $\Delta x=(x_\mathrm{max}-x_\mathrm{min})/2$ and $\Delta y=(y_\mathrm{max}-y_\mathrm{min})/2$ are half the side lengths of a rectangle bounding the modeled film and $(\bar{x}, \bar{y})$ are the coordinates of the center of the rectangle.

The matrix version of Eq.~\ref{eq:Heff_sub} is:

\begin{equation}
    \label{eq:Heff_sub_num}
     \mathbf{h}_{z,\,\mathrm{applied}} - \sum_{\mathrm{holes}\, k}\mathbf{h}_{z,\,\mathrm{eff},\,k} = -(\mathbf{Q}\cdot\mathbf{w}-\mathbf{\Lambda}\cdot\mathbf{\nabla}^2)\mathbf{g},
\end{equation}
where we exclude points in the mesh lying outside of the superconducting film, but keep points
inside holes in the film. $\mathbf{\Lambda}$ is either a scalar or a vector defining the effective penetration depth at every included vertex in the mesh, and $\mathbf{\nabla}^2$
is the Laplace operator (see Section[]), an $n\times n$ matrix defined such that $\mathbf{\nabla}^2\mathbf{f}$ computes the Laplacian $\nabla^2f(x,y)$ of a function $f(x,y)$ defined on the mesh.

Eq.~\ref{eq:Heff_sub_num} is a matrix equation relating the applied field to the stream function
inside a superconducting film, which can efficiently be solved (e.g. by matrix inversion or LU decomposition) for the unknown vector $\mathbf{g}$, the stream function inside the film. Since the stream function outside the film and inside holes in the film is already known, solving Eq.~\ref{eq:Heff_sub_num} gives us the stream function for the full mesh:

\begin{equation}
    \label{eq:full_stream}
    \mathbf{g} = \begin{cases}
        \left(-[\mathbf{Q}\cdot\mathbf{w}-\mathbf{\Lambda}\cdot\mathbf{\nabla}^2]\right)^{-1}
        \left(\mathbf{h}_{z,\,\mathrm{applied}} - \sum_{\mathrm{holes}\,k}\mathbf{h}_{z,\,\mathrm{eff},\,k}\right)
            & \text{inside the film}\\
        I_{\mathrm{circ},\,k}
            & \text{inside hole }k\\
        0
            & \text{elsewhere}
    \end{cases}
\end{equation}

Once the stream function $\mathbf{g}$ is known for the full mesh,
the sheet current flowing in the film can be computed from Eq.~\ref{eq:stream}, the $z$-component of the total field in the plane of the film can be computed
from Eq.~\ref{eq:field_from_kernel_num}, and the full vector magnetic field $\vec{H}(x, y, z)$
at any point in space can be computed from Eq.~\ref{eq:field_from_kernel}.

\subsection{Laplace operator}
\label{section:implementation:laplace}

The definition of the mesh Laplace operator $\mathbf{\nabla}^2$ (also called the Laplace-Beltrami operator [Laplacian-SGP-2014]) deserves special attention, as it reduces the problem of solving a partial differential equation $\nabla^2g(x,y)=f(x,y)$ to the problem of solving a matrix equation
$\mathbf{\nabla}^2\mathbf{g}=\mathbf{f}$. As described in [Vaillant-Laplacian-2013] and [Laplacian-SGP-2014] the Laplace operator $\mathbf{\nabla}^2$ for a mesh is defined in terms of two matrices, the mass matrix $\mathbf{M}$ and the
Laplacian matrix $\mathbf{L}$: $\mathbf{\nabla}^2 = \mathbf{M}^{-1}\mathbf{L}$.


In a 2D mesh, the mass matrix gives an effective area to each vertex in the mesh. There are multiple ways to construct the mass matrix, but here we use a ``lumped" mass matrix, which is diagonal with elements $(\mathbf{M})_{ii} = \frac{1}{3}\sum_{t\in\mathcal{N}(i)}\mathrm{area}(t)$,
where $\mathcal{N}(i)$ is the set of triangles $t$ adjacent to vertex $i$.

Each element $w_{ij}$ of the symmetric weight matrix $\mathbf{w}$ assigns a weight to the edge connecting vertex $i$ and vertex $j$ in the mesh. We use a normalized version of the weight matrix, where the sum of all off-diagonal elements in each row (or column) is 1, and each diagonal element is 1. Thus we can write $\mathbf{w}$ in terms of an un-normalized weight matrix $\mathbf{W}$:
$$
    w_{ij} = \begin{cases}
        1&\text{if }i = j\\
        W_{ij} / \sum_{i\neq j} W_{ij}&\text{otherwise}
    \end{cases}
$$

There are several different methods for constructing the un-normalized weight matrix $\mathbf{W}$:

\begin{enumerate}
    \item{
        Uniform weighting: In this case, $\mathbf{W}$ is simply the adjacency matrix for the mesh:
        $$
            W_{ij} =
            \begin{cases}
                0&\text{if }i=j\\
                1&\text{if }i\text{ is adjacent to }j\\
                0&\text{otherwise}
            \end{cases}
        $$
    }
    \item{
        Inverse-Euclidean weighting: Each edge is weighted by the inverse of its length: $|\vec{r}_i-\vec{r}_j|^{-1}$, where $\vec{r}_i$ is the position of vertex $i$.
        $$
            W_{ij} =
            \begin{cases}
                0&\text{if }i=j\\
                |\vec{r}_i-\vec{r}_j|^{-1}&\text{if }i\text{ is adjacent to }j\\
                0&\text{otherwise}
            \end{cases}
        $$
    }
    \item{
        Half-cotangent weighting: Each edge is weighted by the half the sum of the cotangents of the two angles opposite to it.
        $$
            W_{ij} =
            \begin{cases}
                0&\text{if }i=j\\
                \frac{1}{2}\left(\cot\alpha_{ij}+\cot\beta_{ij}\right)&\text{if }i\text{ is adjacent to }j\\
                0&\text{otherwise}
            \end{cases}
        $$
    }
\end{enumerate}

By default, \SuperScreen uses half-cotangent weighting. The Laplacian matrix $\mathbf{L}$ is defined in terms of the weight matrix $\mathbf{w}$: $(\mathbf{L})_{ij} = (\mathbf{w})_{ij} - \delta_{ij}\sum_{\ell}w_{i\ell}$. Finally, the Laplace operator is given by $\mathbf{\nabla}^2 = \mathbf{M}^{-1}\mathbf{L}$.

\section{Package Overview}
\label{section:overview}

In this section we give a high-level overview of the \SuperScreen package. More detailed documentation and tutorials can be found at superscreen.readthedocs.io.

\subsection{Parameters}
\label{section:overview:parameter}

A \mintinline{python}{superscreen.Parameter} is a wrapper around a Python function that computes a scalar value as a function of position coordinates ($x$, $y$, and optionally $z$) based on some keyword arguments. Addition, subtraction, multiplication, division, and exponentiation between \mintinline{python}{Parameters}, other \mintinline{python}{Parameters}, and Python scalars (\mintinline{python}{int} and \mintinline{python}{float}) are supported. For example:

\begin{minted}{python}
from superscreen import Parameter

def f(x, y, a=1, b=0):
    return a * x**2 + b * y
    
param = Parameter(f, a=0, b=1)
print(param)  # Parameter<f(a=0, b=1)>

other_param = Parameter(f, a=2, b=3)
print(other_param)  # Parameter<f(a=2, b=3)>

print(param(0, 0) == f(0, 0, a=0, b=1))  # True
print(other_param(1, 2) == f(1, 2, a=2, b=3))  # True

param_squared = param ** 2
print(param_squared)  # CompositeParameter<(f(a=0, b=1) ** 2)>

print(param_squared(3, 4) == f(3, 4, a=0, b=1) ** 2) # True
print(
    (5 * param_squared / other_param)(5, 6)
    == 5 * f(5, 6, a=0, b=1) ** 2 / f(5, 6, a=2, b=3)
)  # True
\end{minted}

\mintinline{python}{Parameters} can be used to define applied magnetic fields and penetration depths that depend on position.

\subsection{Devices}
\label{section:overview:device}

Information about the geometry and penetration depth of a superconducting structure is described by an instance of the \mintinline{python}{superscreen.Device} class. A \mintinline{python}{Device} is made up of one or more superconducting layers, each an instance of  \mintinline{python}{superscreen.Layer}. Each layer sits in a specified plane parallel to the $x-y$ plane and has its own effective penetration depth $\Lambda$, which can either be a constant or a \mintinline{python}{superscreen.Parameter} wrapping a function that computes $\Lambda(x, y)$, the penetration depth as a function of position. The effective penetration depth $\Lambda$ can instead be defined in terms of a layer's London penetration depth $\lambda$ and its thickness $d$: $\Lambda=\lambda^2/d$, in which case the London penetration depth can be either a constant or a \mintinline{python}{superscreen.Parameter}.

Each layer can contain one or more superconducting films which may have one or more holes in them. Films and holes are represented by instances of the \mintinline{python}{superscreen.Polygon} class. A \mintinline{python}{Polygon} contains the name of the layer to which is belongs, an array specifying the coordinates of its vertices, and a method, \mintinline{python}{Polygon.contains_points()}, which determines whether specified coordinates lie within the polygon or not. In addition to superconducting films and holes, one may define ``abstract regions,'' which are polygons that need not correspond to a physical feature in the structure, but will still be meshed. Abstract regions can be used to define a ``bounding box" around a structure.

Once the layers, films, holes, and abstract regions have been defined, one can generate the computational mesh by calling \mintinline{python}{Device.make_mesh()}. The region that is meshed is defined by the convex hull of the union of all polygons in the device. After the mesh is generated, the matrices and vectors described in Section~\ref{section:implementation} can be computed and one can begin solving models.

\subsection{Solvers}
\label{section:overview:solvers}

A \SuperScreen model consists of a \mintinline{python}{Device}, a function or \mintinline{python}{Parameter} that computes the applied magnetic field as a function of position $H_{z,\,\mathrm{applied}}(x, y, z)$, and (optionally) a value for the current circulating around each hole in the device due to trapped flux. These items serve as the inputs to \SuperScreen's main solver function, \mintinline{python}{superscreen.solve()}, which implements the calculation outlined in Section~\ref{section:implementation}. When simulating a device with more than one layer, one can specify the number of times to implement the iterative calculation described in Section~\ref{section:model:multilayer} in order to solve for the response of all layers self-consistently. One can also skip the iterative portion of the calculation entirely and only solve for the response of each layer to the applied field, assuming interaction between layers.

The output of \mintinline{python}{superscreen.solve()} is a \mintinline{python}{list} of \mintinline{python}{superscreen.Solution} objects, with a length of 1 plus the number of iterations used for the iterative portion of the calculation. A \mintinline{python}{Solution} encapsulates all of the information about a solved model: the \mintinline{python}{Device}, applied field, circulating currents, and calculated stream functions and screening fields for all layers in the device. A \mintinline{python}{Solution} also has methods for processing the simulation results, including:
\begin{itemize}
    \item{
    \mintinline{python}{Solution.grid_data()}: Interpolates the calculated stream functions and magnetic fields for each layer from the triangular mesh to a rectangular grid.
    }
    \item{
    \mintinline{python}{Solution.current_density()}: Interpolates the stream functions to a rectangular grid and calculates the current density (sheet current) $\vec{J}(x, y)$ in each layer using Eq.~\ref{eq:stream}.
    }
    \item{
    \mintinline{python}{Solution.polygon_flux()}: Calculates the total flux through each polygon in the device (films, holes, and abstract regions).
    }
    \item{
    \mintinline{python}{Solution.field_at_position()}: Calculates the vector magnetic field at any point(s) in space due the applied field and the currents flowing the in the device using Eqs.~\ref{eq:field_from_kernel} and \ref{eq:kernels}.
    }
\end{itemize}

One may wish to solve many models involving the same device while varying other aspects of the model, for example sweeping the applied field, circulating currents, or some parameter of one or more layers in the device. Fortunately, the mesh and large matrices described in Section~\ref{section:implementation} depend only on the geometry of the device parallel to the $x-y$ plane. This means that the same mesh and matrices can be re-used for models with different applied fields, circulating currents, layer heights, and layer penetration depths.

The \mintinline{python}{superscreen.solve_many()} function is designed to do just that. One can provide a sequence of \mintinline{python}{superscreen.Parameter} objects defining different applied fields and/or a sequence of circulating current values over which to sweep. One can also provide a ``layer updater" function that modifies each layer in the device according to some set of keyword arguments, which can also be swept. The latter option can be used to sweep layer heights or penetration depths. Given these inputs, \mintinline{python}{superscreen.solve_many()} will generate and solve all of the corresponding models. The models can either be solved in series in a single Python process (the default), or in parallel across multiple Python processes running on different CPUs. In either case, only a single copy of the device's mesh and large arrays is created, so the memory footprint is modest.

\subsection{Visualization}
\label{section:overview:visualization}

\SuperScreen offers several functions for visualizing the results of simulations (which are also aliased as method on \mintinline{python}{superscreen.Solution}):

\begin{itemize}
    \item{
    \mintinline{python}{superscreen.plot_streams()}: Given a \mintinline{python}{Solution}, plots the stream function $g(x, y)$ for one or more layers in the device. See also \mintinline{python}{Solution.plot_streams()}.
    }
    \item{
    \mintinline{python}{superscreen.plot_currents()}: Given a \mintinline{python}{Solution}, plots the sheet current $\vec{J}(x, y)$ for one or more layers in the device. See also \mintinline{python}{Solution.plot_currents()}.
    }
    \item{
    \mintinline{python}{superscreen.plot_fields()}: Given a \mintinline{python}{Solution}, plots the total field $H_z(x, y)$ or the screening field $H_z(x, y) - H_{z,\,\mathrm{applied}}(x, y)$ for one or more layers in the device. See also \mintinline{python}{Solution.plot_fields()}.
    }
    \item{
    \mintinline{python}{superscreen.plot_field_at_positions()}: Given a \mintinline{python}{Solution}, plots the total field $\vec{H}(x, y, z_0)$ or $H_z(x, y, z_0)$ at any points in space (where $z_0$ is not equal to the vertical position of any of the layers in the device). See also \mintinline{python}{Solution.plot_field_at_positions()}.
    }
\end{itemize}

See Section~\ref{section:examples} for examples of the figures generated by these functions.

\subsection{Comparison \& Persistence}
\label{section:overview:persistence}

\mintinline{python}{Parameters}, \mintinline{python}{Layers}, \mintinline{python}{Polygons}, \mintinline{python}{Devices}, and \mintinline{python}{Solutions} all implement the equality operator, \mintinline{python}{==}. Two \mintinline{python}{Parameters} are considered equal if the Python bytecode of their underlying functions is the same and their keyword arguments are the same. Two \mintinline{python}{Layers} are equal if their name, penetration depth, thickness, and vertical position are all equal. Two \mintinline{python}{Polygons} are equal if they are in the same layer and their name and polygon vertices are equal. Two \mintinline{python}{Devices} are equal if their name, layers, films, holes, and abstract regions are all equal. Two \mintinline{python}{Solutions} are equal if their device, applied field, circulating currents, timestamp (time at which the solution was created), and all stream function and magnetic field arrays are equal. Two solutions created at different times can also be compared by calling \mintinline{python}{solution.equal(other_solution, require_same_timestamp=False)}.

Instances of \mintinline{python}{superscreen.Device} and \mintinline{python}{superscreen.Solution} can be saved to and loaded from disk using their respective \mintinline{python}{to_file()} and \mintinline{python}{from_file()} methods, making it straightforward to share models and simulation results. \mintinline{python}{Layers}, \mintinline{python}{Polygons}, and all metadata are serialized to JSON, a widely-used, human-readable plain text format. Functions and \mintinline{python}{Parameters}, such as those that compute the applied field or penetration depth, are serialized in binary form using \mintinline{python}{dill}. \mintinline{python}{numpy} arrays, such as the mesh itself and the computed stream functions and fields, are saved in the well-established \mintinline{python}{numpy} \mintinline{python}{npz} file format. A \mintinline{python}{list} of \mintinline{python}{Solutions}, such as that returned by \mintinline{python}{superscreen.solve()} can be saved/loaded all at once using \mintinline{python}{superscreen.save_solutions()}/\mintinline{python}{superscreen.load_solutions()}.

\subsection{Development Details}
\label{section:overview:development}

\SuperScreen requires Python version 3.6 or later. The package is located in a public repository on GitHub, and a suite of unit tests is run automatically via GitHub Actions whenever a change or proposed change (Pull Request) is made to the \mintinline{python}{main} branch of the repository. At the time of writing, the test suite is executed using Python versions 3.6, 3.7, 3.8, and 3.9, and the test coverage is XX\%. Any changes to the \mintinline{python}{main} branch of the repository also trigger an automatic re-build of the online documentation, hosted by Read the Docs at superscreen.readthedocs.io. Stable versions of the package are tagged on GitHub and uploaded to PyPI, the Python Package Index. The source code and documentation are covered by the MIT License.

\SuperScreen has several important dependencies beyond the Python standard library: \mintinline{python}{numpy} and \mintinline{python}{scipy} for numerics, \mintinline{python}{pint} for handling physical units, \mintinline{python}{matplotlib} for visualization, \mintinline{python}{meshpy} and \mintinline{python}{optimesh} for mesh generation, \mintinline{python}{dill} for serializing objects to disk and between processes, and \mintinline{python}{ray} for parallel processing with shared memory.

\section{Examples}
\label{section:examples}

\subsection{Screening}
\label{section:example:screening}

\subsection{Inductance}
\label{section:examples:inductance}

\subsection{Mutual Inductance}
\label{section:examples:mutual-inductance}

\subsection{SQUID Susceptometry}
\label{section:examples:susceptometry}

\subsection{Parallel Processing}
\label{section:examples:parallel}


\section{Conclusion}
\label{section:conlusion}

\end{document}
