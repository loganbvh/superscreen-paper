\documentclass{article} % This command is used to set the type of document you are working on such as an article, book, or presenation

\usepackage{geometry} % This package allows the editing of the page layout
\usepackage{amsmath}  % This package allows the use of a large range of mathematical formula, commands, and symbols
\usepackage{graphicx}  % This package allows the importing of images

\title{Response to reviewer comments}
\author{Logan Bishop-Van Horn}
\date{\today}

\begin{document}
\maketitle

\section{Reviewer 1}

\begin{itemize}
    \item{
    {\bf Reviewer comment:} ``The manuscript is well written and the work is of interest to the applied superconductivity community.''
    
    {\bf Author response:} We are glad that the reviewer found the paper to be well written, and we agree that the software and manuscript will be valuable to the applied superconductivity community.
    }
    \item{
    {\bf Reviewer comment:} ``The software works, although the implementation is a bit slower and more memory hungry than commercial or closed-source tools.''
    
    {\bf Author response:} It is true that the software favors portability and simplicity of implementation over raw performance, likely making it less efficient than some commercial or closed-source tools. We do not see SuperScreen as a competitor to or replacement for superconducting electronic design automation (EDA) and inductance extraction tools such as InductEx and FastHenry. Rather, we see SuperScreen as a complementary research tool for modeling experiments involving 2D superconductors and mesoscopic superconducting devices, as opposed to designing superconducting integrated circuits. 
    }
    \item{
    {\bf Reviewer comment:} ``The reference to [16] for a review of inductance extraction is a bit old -- newer tools and methods and publications exist.''
    
    {\bf Author response:} Reference 16 is indeed a bit old. We have added more recent references
    }
    \item{
    {\bf Reviewer comment:} ``P.4, Left Column, Line 53: `though' should be `through'.''
    
    {\bf Author response:} Thank you for the careful reading and for catching this typo. The typo has been fixed in the revised version of the manuscript.
    }
\end{itemize}

\section{Reviewer 2}
\begin{itemize}
    \item{
    {\bf Reviewer comment:} ``Highlights and graphical abstract could be provided.''
    
    {\bf Author response:}
    }
    \item{
    {\bf Reviewer comment:} ``As there are many other similar works available, please provide a strong literature survey.''
    
    {\bf Author response:}
    }
    \item{
    {\bf Reviewer comment:} ``A comparison of the work with state of the art is necessary.''
    
    {\bf Author response:}
    }
\end{itemize}

\end{document}