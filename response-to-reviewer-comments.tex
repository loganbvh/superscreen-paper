\documentclass{article} % This command is used to set the type of document you are working on such as an article, book, or presenation

\usepackage{geometry} % This package allows the editing of the page layout
\usepackage{amsmath}  % This package allows the use of a large range of mathematical formula, commands, and symbols
\usepackage{graphicx}  % This package allows the importing of images

\title{Response to reviewer comments}
\author{Logan Bishop-Van Horn}
\date{\today}

\begin{document}
\maketitle

\section{Reviewer 1}

\begin{itemize}
    \item{
    {\bf Reviewer comment:} ``The manuscript is well written and the work is of interest to the applied superconductivity community.''
    
    {\bf Author response:} We are glad that the reviewer found the paper to be well written, and we agree that the software and manuscript will be valuable to the applied superconductivity community.
    }
    \item{
    {\bf Reviewer comment:} ``The software works, although the implementation is a bit slower and more memory hungry than commercial or closed-source tools.''
    
    {\bf Author response:} It is true that the software favors portability and simplicity of implementation over raw performance, likely making it less efficient than some commercial or closed-source tools. We do not see SuperScreen as a competitor to or replacement for superconducting electronic design automation (EDA) and inductance extraction tools such as InductEx and FastHenry. Rather, we see SuperScreen as a complementary research tool for modeling experiments involving 2D superconductors and mesoscopic superconducting devices, as opposed to designing superconducting integrated circuits. 
    }
    \item{
    {\bf Reviewer comment:} ``The reference to [16] for a review of inductance extraction is a bit old -- newer tools and methods and publications exist.''
    
    {\bf Author response:} Reference [16] (Gaj, \emph{et al.}) is indeed a bit old. We have added more recent references about inductance extraction tools to Appendix A. In particular, Tolpygo, \emph{et al.} 2021 Supercond. Sci. Technol. {\bf 34} 085005 (which is now cited as reference [20]) and references [8--25] therein provide a thorough and up-to-date survey of inductance extraction software.
    }
    \item{
    {\bf Reviewer comment:} ``P.4, Left Column, Line 53: `though' should be `through'.''
    
    {\bf Author response:} Thank you for reading the manuscript carefully and catching this typo. The typo has been fixed in the revised version of the manuscript.
    }
\end{itemize}

\section{Reviewer 2}
\begin{itemize}
    \item{
    {\bf Reviewer comment:} ``Highlights and graphical abstract could be provided.''
    
    {\bf Author response:} Thank you for the suggestion. We have included highlights and a graphical abstract in the revised submission.
    }
    \item{
    {\bf Reviewer comment:} ``As there are many other similar works available, please provide a strong literature survey.''
    
    {\bf Author response:} The introductory section provides a survey of the literature of numerical modeling of 2D superconductors (references 1--20). In terms of surveying available software tools, we have added more recent references about inductance extraction tools to Appendix A. In particular, Tolpygo, \emph{et al.} 2021 Supercond. Sci. Technol. {\bf 34} 085005 (which is now cited as reference [20]) and references [8--25] therein provide a thorough and up-to-date survey of inductance extraction software.
    }
    \item{
    {\bf Reviewer comment:} ``A comparison of the work with state of the art is necessary.''
    
    {\bf Author response:} Figure 4 provides a comparison between SuperScreen, 3D-MLSI, LCR2D, and the axisymmetric model by  Brandt and Clem. The problem shown in Figure 4 (the self-inductance of an annulus as a function of the effective penetration depth $\Lambda$, inner radius $a$, and outer radius $b$) is the only problem for which we were able to find solutions from multiple methods in the literature. This problem also has an analytical solution in the limit $\Lambda=0$ and $a/b\to 0$. All of the numerical methods, including SuperScreen, agree well with the analytical solution in this limit.
    
    To the best of our knowledge, the ``state of the art'' commercial tool for modeling the magnetic response of superconductors is InductEx. We have added to Appendix A an example problem from the InductEx user manual, namely the self-inductance of a thin film square washer. Code Block 3 in Appendix A, which we have added in the revised version of the manuscript, demonstrates that SuperScreen and InductEx agree to within 1\% on this problem.
    }
\end{itemize}

\section{Inhomogeneous films}

After submitting the manuscript and posting the submitted version to arXiv.org, Dr. Vladimir G. Kogan brought to our attention his work on London modeling of superconductors with inhomogeneous penetration depth (V. G. Kogan \& J. R. Kirtley, Phys. Rev. B {\bf 83}, 214521 (2011)) and pointed out the existence of an additional term in the 2nd London equation for inhomogeneous superconductors, which is proportional to the gradient of the effective penetration depth, $\vec{\nabla}\Lambda$.

We have modified the software so that this term is included if the user specifies an inhomogeneous penetration depth and have moved all discussion of inhomogeneous films from the main text into Appendix C, where we have clarified the applicability of the London model to films with inhomogeneous penetration depth based on V. G. Kogan \& J. R. Kirtley, Phys. Rev. B {\bf 83}, 214521 (2011) and J. R. Cave \& J. E. Evetts, Journal of Low Temperature Physics {\bf 63}, 35–55 (1986) (references [52] and [51] respectively in the revised manuscript). We also removed from Appendix C the example code (Code Block 3: Simulating an inhomogeneous superconducting device) and the accompanying Figure C.7. The main results of the manuscript are completely unchanged.
\end{document}